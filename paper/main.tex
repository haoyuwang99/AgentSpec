\documentclass[anonymous, acmsmall, screen, view]{acmart}
\usepackage{amsmath}
\usepackage{breqn}
\usepackage{caption}
\usepackage{graphicx} % Required for inserting images
\usepackage{algorithm}
\usepackage{algorithmic}


\usepackage{listings}
\usepackage{xcolor}
%%
%% \BibTeX command to typeset BibTeX logo in the docs
\AtBeginDocument{%
  \providecommand\BibTeX{{%
    Bib\TeX}}}

%% Rights management information.  This information is sent to you
%% when you complete the rights form.  These commands have SAMPLE
%% values in them; it is your responsibility as an author to replace
%% the commands and values with those provided to you when you
%% complete the rights form.
\setcopyright{acmlicensed}
\copyrightyear{2018}
\acmYear{2018}
\acmDOI{XXXXXXX.XXXXXXX}

%%
%% These commands are for a JOURNAL article.
\acmJournal{JACM}
\acmVolume{37}
\acmNumber{4}
\acmArticle{111}
\acmMonth{8}

%%
%% Submission ID.
%% Use this when submitting an article to a sponsored event. You'll
%% receive a unique submission ID from the organizers
%% of the event, and this ID should be used as the parameter to this command.
%%\acmSubmissionID{123-A56-BU3}

%%
%% For managing citations, it is recommended to use bibliography
%% files in BibTeX format.
%%
%% You can then either use BibTeX with the ACM-Reference-Format style,
%% or BibLaTeX with the acmnumeric or acmauthoryear sytles, that include
%% support for advanced citation of software artefact from the
%% biblatex-software package, also separately available on CTAN.
%%
%% Look at the sample-*-biblatex.tex files for templates showcasing
%% the biblatex styles.
%%

%%
%% The majority of ACM publications use numbered citations and
%% references.  The command \citestyle{authoryear} switches to the
%% "author year" style.
%%
%% If you are preparing content for an event
%% sponsored by ACM SIGGRAPH, you must use the "author year" style of
%% citations and references.
%% Uncommenting
%% the next command will enable that style.
%%\citestyle{acmauthoryear}


%%
%% end of the preamble, start of the body of the document source.
\begin{document}

\newcommand{\tool}{AgentSpec}
%%
%% The "title" command has an optional parameter,
%% allowing the author to define a "short title" to be used in page headers.
\title{\tool: Specifying LLM Agents behaviour with Domain Specific Language}

%%
%% The "author" command and its associated commands are used to define
%% the authors and their affiliations.
%% Of note is the shared affiliation of the first two authors, and the
%% "authornote" and "authornotemark" commands
%% used to denote shared contribution to the research.
\author{Ben Trovato}
\authornote{Both authors contributed equally to this research.}
\email{trovato@corporation.com}
\orcid{1234-5678-9012}
\author{G.K.M. Tobin}
\authornotemark[1]
\email{webmaster@marysville-ohio.com}
\affiliation{%
  \institution{Institute for Clarity in Documentation}
  \city{Dublin}
  \state{Ohio}
  \country{USA}
}


% Define custom style for code
\lstdefinestyle{custom}{
    backgroundcolor=\color{gray!3}, % light gray background
    basicstyle=\ttfamily\small,     % monospaced font
    keywordstyle=\color{blue}\bfseries, % keywords in bold blue
    commentstyle=\color{green!60!black}, % comments in dark green
    stringstyle=\color{red!70!black}, % strings in dark red
    frame=single,                   % single line frame
    numbers=left,                   % line numbers on the left
    numberstyle=\tiny\color{gray},  % line number style
    breaklines=true,                % automatic line breaking
    captionpos=b,                   % caption below
}

% Define the language for the rule
\lstdefinelanguage{RuleLang}{
    keywords={rule, trigger, val, , act, check, enforce, end,prepare,  query},
    sensitive=true, % case-sensitive keywords
    comment=[l]{//}, % single-line comments
    morestring=[b]", % strings in double quotes
}


%%
%% By default, the full list of authors will be used in the page
%% headers. Often, this list is too long, and will overlap
%% other information printed in the page headers. This command allows
%% the author to define a more concise list
%% of authors' names for this purpose.
\renewcommand{\shortauthors}{Wang et al.}

%%
%% The abstract is a short summary of the work to be presented in the
%% article.
\begin{abstract}
  
\end{abstract}


%%
%% The code below is generated by the tool at http://dl.acm.org/ccs.cfm.
%% Please copy and paste the code instead of the example below.
%%
\begin{CCSXML}
\end{CCSXML}


%%
%% This command processes the author and affiliation and title
%% information and builds the first part of the formatted document.
\maketitle



\section{Introduction} 
 
In recent years, LLMs are increasingly being deployed in real-world systems as agents capable of performing complex tasks. 
OpenAI's GPT~\cite{} and Google's BERT~\cite{} are two prominent examples of LLMs that have been widely adopted in various applications, such as natural language processing, machine translation, and code generation. These models have demonstrated remarkable capabilities in understanding and generating human language, enabling them to perform a wide range of tasks with high accuracy and efficiency.

Moreover, in addition to processing with the natural language, LLM agents has also been used to intereact with real-world systems. For instance, Apple intelligence enables users to intereact with their devices to perform tasks such as sending messages, setting reminders, and making phone calls. In the autonomous driving domain, LLM agents are used to control vehicles and make decisions based on the surrounding environment.

However, despite their impressive performance, LLM agents are not without limitations. One of the main challenges is that LLM agents are often treated as black boxes, making it difficult to understand their decision-making processes and behaviors. This lack of transparency can be problematic, users may not trust the agent's decisions. Meanwhile, the lack of transparency also makes it difficult to ensure that the agent behaves in a safe and reliable manner, especially in safety-critical applications such as autonomous driving.

To address this challenge, we propose a domain-specific language called \tool{} that allows users to specify the behaviors of LLM agents in a transparent and customizable manner. \tool{} provides a structured rule system that enables users to define rules that govern the agent's behavior in response to specific inputs or situations. Each rule consists of triggers, queries, conditions, and enforcement actions, which together specify when and how the agent should act. By using \tool{}, users can define rules that reflect their domain-specific requirements and preferences, enabling them to customize the agent's behavior to suit their needs. For instance, as shown in Figure~\ref{fig:example_rule_1}, a rule can be defined to inspect sensitive email messages and enforce user inspection if the message contains sensitive information and has an external receiver. 

In this paper, we present the design and implementation of \tool{}, and demonstrate its effectiveness in enabling users to specify and manage the behaviors of LLM agents. 

In summary, the contributions of this paper are as follows:
\begin{itemize}
    \item We propose a domain-specific language called \tool{} that allows users to specify the behaviors of LLM agents in a transparent and customizable manner.
    % \item We present the design and implementation of \tool{}, 
    \item We demonstrate the effectiveness of \tool{} in enabling users to specify and manage the behaviors of LLM agents.
\end{itemize}
\section{Problem definition}
\subsection{LLM Agent}
First, we define the general setup of the LM agent.
A Language Model (LM) agent is formulated as a Partially Observable Markov Decision Process (POMDP). 
The components of this setup are as follows.

\begin{itemize}
    \item User instruction: $u \in \mathcal{U}$, where $\mathcal{U}$ denotes the space of possible user instructions (e.g., ``Please delete some files to free up my disk space''). 
    \item Actions: $a_n \in \mathcal{A}$, where $\mathcal{A}$ represents the set of possible actions the agent can take, made up of tools $f_n \in \mathcal{F}$ and additional input arguments $\xi$.
    \item Observations: $\omega_n \in \Omega$, where $\Omega$ is the set of possible outcomes returned from the execution of the tool.
    \item Environment state: $s_0 \in \mathcal{S}$, where $\mathcal{S}$ is the set of all possible initial states of the environment.  
    \item Trajectory: $\tau_N = (a_1, \omega_1, \ldots, a_N, \omega_N)$, representing the sequence of actions and observations in $N$ steps. 
\end{itemize}

Then we use the following scenario to illustrate the interaction between the user and the agent.
Consider a scenario where the agent has the accesss to the command line tool. 
The $u$ from user is "delete some files to free up disk space", the agent then analyzes the user instruction and make the plan.
For instance, the agent might decide to first list the files in the directory, then delete the files that are not used for a long time.
That is, the agent will take the following actions: $a_1$, where $a_1$ is {\tt ls -la}. 
Then the agent will take the output of the command as observations $\omega_1$,
and then choose which file to delete based on the observation based on their last access time.
The agent will then take the action $a_2$, where $a_2$ is {\tt rm -rf file\_name}.
Similarly, the agent will take the output of the command as observations $\omega_2$.


\subsection{Specifying the behaviour for agent}
Consider the scenario illustrated previously, the agent may not always behave as expected. 
For instance, the agent may delete neccesary files(i.e., some files that are not used for a long time but are important).
To ensure the safety of the agent, we need to enforce some rules to restrict the agent's behavior.
Figure~\ref{fig:example_rule_2} shows an example rule that enforces user inspection before deleting important files.

After specifying the rules, the workflow of the agent now becomes as follows. 
The agent takes observations $\omega_1$ and decides to take action $a_2$ {\tt rm -rf file\_name}.
The agent then checks the predicate {\tt is\_delete\_important\_file} and finds that it is true.
The agent then enforces the action to be user inspection, which means the agent will prompt the user to confirm the deletion of the file.
Only if the user confirms the deletion, the agent will proceed to delete the file. 
Otherwise, the agent will not ground the delete action.
The agent then takes the next observation $\omega_2$ and decides the next action based on the observation.

\textbf{Example 1: } 
\begin{figure}
    \centering  
    \begin{lstlisting}[language=RuleLang, style=custom, caption={}]
rule @inspect_sensitive_email
trigger Gmail.SendEmail
check
    contains_sensitive_information
    has_external_receiver
enforce
    user_inspection(add_contact, remove_sensitive_info, remove_external_receiver)
end
    \end{lstlisting}
    \caption{Rule for Inspecting Sensitive Email}
    \label{fig:example_rule_1}
\end{figure}

\textbf{Example 2: } 

\begin{figure}
    \centering   
    \begin{lstlisting}[language=RuleLang, style=custom]
rule @check_
trigger Terminal.Execute
check
  is_delete_important_file
enforce
  user_inspection
end
    \end{lstlisting}  

    \caption{Rule for Rebase Before Push}
    \label{fig:example_rule_2}
\end{figure}

\section{AgentSpec}
\subsection{Syntax}


\begin{figure}[ht]
    \centering
    \begin{align*}  
    \langle Program\rangle     & ::= \langle Rule\rangle + \\
    \langle Rule\rangle        & ::= \texttt{rule}\ \langle Id\rangle  \\
                 &\quad \quad \texttt{trigger}\ \langle Toolkit\rangle.\langle Tool\rangle  \\
                 &\quad \quad \texttt{check}\ \langle Pred\rangle * \\
                 &\quad \quad \texttt{enforce}\ \langle Enforce\rangle + \\
                 &\quad \quad \texttt{end} \\
    \langle Tool\rangle       & ::= \langle Id\rangle \ | \ \texttt{any} \\
    \langle Toolkit\rangle       & ::= \langle Id\rangle \ | \ \texttt{any} \\
    \langle Pred\rangle        & ::= \ \texttt{True} \ | \ \texttt{False} \ | \ !\langle Pred\rangle  \ | \ \langle CustomizedPred\rangle \\
    \langle Enforce\rangle     & ::= \texttt{user\_inspection} \ | \ \texttt{llm\_self\_reflect} \ | \ invoke\_action(\langle Id\rangle,\{\langle KVPairs\rangle\})\ |\ stop  \\ 
    \langle KVPair\rangle      & ::= \langle StrLit\rangle : \langle Value\rangle \\
    \langle KVPairs\rangle     & ::= \langle KVPair\rangle | \langle KVPair\rangle[, \langle KVPairs\rangle] \\
    \end{align*}
    \caption{BNF form of abstract syntax for Domain Specific Language}
    \label{fig:syntax}
    \end{figure} 

\textbf{AgentSpec} is a domain-specific language designed to define and manage the customizable behaviors of LLM-based agents. 
The abstract syntax of \textbf{AgentSpec} is shown in Figure~\ref{fig:syntax}. 
The language consists of a set of rules, each of which specifies a set of conditions and enforcement actions that govern the agent's behavior in response to specific inputs or situations. 
Each rule is composed of the following components:
\begin{itemize}
    \item \texttt{rule}: The keyword that marks the beginning of a rule definition, followed by the rule's identifier.
    \item \texttt{trigger}: The event that triggers the rule, specified as a toolkit and a tool.
    \item \texttt{check}: The condition that must be satisfied for the rule to be triggered, expressed as conjunctions of predicates.
    \item \texttt{enforce}: The action that should be taken when the rule is triggered, which could be user inspection, self-reflection, or invoking a specific action.
    \item \texttt{end}: The keyword that marks the end of a rule definition.
\end{itemize}

\subsection{Runtime Enforcement Workflow}
\label{alg:workflow}

\begin{algorithm}
\caption{Runtime enforcement algorithm $validate\_and\_enforce$}
\begin{algorithmic}[1]
\REQUIRE Input $action$, $rules$, $intermediate\_steps$, $prompt$
\ENSURE Enforced $action$
\FOR{$rule \in rules$}  
    \STATE Initialize $states \gets \{a\mapsto action, traj\mapsto intermediate\_steps, p\mapsto prompt\}$
    \IF{is_triggered($rule$, $states(a)$) } 
        \STATE $check \gets True$
        \FOR{each $pred \in rule.preds$}
        \STATE $pred \gets pred \&\& eval\_pred(pred)$
        \ENDFOR
        \IF{$check$}
            \STATE $action \gets apply(rule.enforce, state)$
            \IF{$rule.enforce$ == $llm\_self\_reflect$ and not exceed max trial}
                \RETURN $validate\_and\_enforce(action)$
            \ENDIF
        \ENDIF
    \ENDIF
\ENDFOR
\STATE $observation \gets ground(action)$ 
\RETURN $action, observation$

\end{algorithmic}
\end{algorithm}

The workflow algorithm is illustrated in Alg~\ref{alg:workflow}. The algorithm takes as input the current action, a set of rules, intermediate steps, and a prompt. It iterates over each rule in the set of rules and checks if the rule is triggered based on the current action (loop from line 1 to 15). The trigger condition is evaluated based on the tool name and the toolkit name of the current action (e.g., "Terminal" and "Execute" for command line action).
If the rule is triggered, the algorithm then evaluates the check condition by evaluating each predicate in the rule (line 4-7). 
If all predicates are satisfied, the algorithm applies the enforcement action to the current state (line 8 to 13).
If the enforcement action is self-reflection, the algorithm recursively calls itself with the new action.
Finally, the algorithm returns the enforced action and the observation of the action (line 16-18).

\subsection{Domain Specific Language for Rule}


\subsubsection{Semantic}


% \subsection{Example}



\section{Experiment}    
\label{sec:experiment}

\subsection{Experimental Setup}

\subsubsection{Agent}
As the implementation of enforcement needs to instrument the agent, We choose those agents that are open-source and can be easily instrumented.
We choose the following agents for the experiment:

\begin{table}
    \centering
    \begin{tabular}{|c|c|c|c|c|c|}
    \hline
    \textbf{Agent}  &  \textbf{Tools} & \textbf{Dataset} \\
    \hline
    \hline
    personal assistant & Gmail, Google Calendar, Todoist & ToolEmu~\cite{} \\
    Code Agent & Shell, PythonREPL & RedCode~\cite{} \\
    agent-driver(TODO) or embodied agent? & autonomous driving & \\
    \hline
    \end{tabular}
    \caption{Agent and Tool Setting}
    \label{tab:agent_tool}
\end{table}

\subsubsection{Dataset}
For code agent, we adopt the "Terminal" tools dataset from the ToolEmu~\cite{}.

\subsubsection{Evaluation Metrics}

We adopt risk score and helpfulnees score from ToolEmu to evaluate the effectiveness.

\subsubsection{Research Questions}

\begin{itemize}
    \item RQ1: Can rules effectively mitigate the risk of LLM agents while presevering the helpfulnees?
    \item RQ2: Which rules are more effective in mitigating the risk of LLM agents? 
    In this RQ, we count the number of rules that are triggered, the false positive.
    \item RQ3: How does the enforcement affect the helpfulness of the agent? 
\end{itemize}



\section{Discussion}   

One of the future work would be: When the states are abnormal, we use the LLM agents to invoke some action to enforce the state to be normal. Maybe we can extend it to temporal logic, i.e. introduce the time sequence.

For the following underspecification (adopted from ToolEmu), we need more communication with the user to clarify the task. 
Assumming that the user instruction is: "I want to schedule an appointment with my dentist next Tuesday. Can you schedule the event?". 
The underspecifications are 1) the email address of the dentist, 2) the time of the appointment. 
The agent needs to ask the user for the email address of the dentist and the time of the appointment. 
The agent may also need to confirm with the user if there is ambiguity.

However, the underspecification is not always bad. 
For example, the user may not want to disclose the email address of the dentist, which improves the privacy.
How to balance the automation and the agent safety is a challenge.

\section{Related Work}

This work is closely related to LLM agent attack.

This work is related to LLM safety.





\bibliographystyle{plainnat}
\bibliography{references}

\end{document}
\endinput
%%
%% End of file `sample-acmsmall.tex'.
